Štvornohé roboty, sú trieda mobilných robotických zariadení využívajúce aproximácie bioligických končatín pre pohyb v priestore.\newline
Oproti kolesovým a pásovým podvozkom je ich spôsob pohybu o mnoho viac flexibilný v terénoch skutočného sveta, čo im umožňuje prekonať zložitejšie prekážky, za ceny vyššej zložitosti kinematiky pohybových akčných členov.\newline
Štvornohý robot Artaban od spoločnosti Panza Robotics využíva túto výhodu pre vonkajšie alebo komplexné priemyslené priestory.\newline\newline

\noindent Ako všetky mobliné roboty, aj tieto musia pri autonómnych aplikáciach riešiť otázku `` Kde som? ''.\newline
Riešením tejto otázky vzniká koncept lokalizácie, ktorá v kažkom snímanom časovom okamihu určuje pozíciu robota. Lokalizovať robot je možné pomocou snímačov velčín vedúcim ku zmene polohy robota (napr. akcelerometer, gyroskop, enkóder uhlovej rýchlosti motora, kamera).\newline
Ak sa v prostredí nachádzajú prekážky, ktoré sú senzory schopné zachytiť, dokážeme túto informáciu uložiť, čím vytvárame mapu prostredia.\newline
\acrshort{slam} je zlúčenie lokalizácie a mapovania, kde sa obe operácie vykonávajú naraz.
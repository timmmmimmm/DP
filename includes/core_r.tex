\section{Artaban}
\begin{center}
    W.I.P
\end{center}

\section{Artaban a SLAM}
Algoritmy \acrshort{slam} vychádzajú zo známeho kinematického modelu mobilného robota, ktorý začína obvykle v neznámej polohe a pohybuje sa v prostredí, ktoré môže obsahovať prirodzené alebo umelé značky prostredia. Úlohou \acrshort{slam}u teda je zostaviť mapu prostredia a zároveň ju použiť na vypočet mobilného robota v reálnom čase.\cite{fs1}\newline
Všeobecne sa dá \acrshort{slam} matematicky opísať ako výpočet distribúcie pravdepodobností:
\begin{equation}
    P(x_k,m_k|Z_{0:k},U_{0:k},x_0)
\end{equation}
Kde:
\begin{itemize}
    \item $x_k$ je aktuálna poloha robota
    \item $m_k$ je aktuálna mapa prostredia
    \item $Z_{0:k}$ sú značky prostredia detekované snímačom
    \item $U_{0:k}$ je priebeh akčných zásahov
    \item $x_0$ je predchádzajúca (apriórna) poloha robota
\end{itemize}
Všeobecné riešenie \acrshort{slam}u je založené na rekurzívnom Bayersovskom odhade s využitím akčného zásahu $u_k$ a pozorovania $z_k$ pomocou ktorého sa aktualizuje aposteriórna pravdepodobnosť s apriórnym odhadom z kroku k-1.\cite{fs2}
\begin{equation}
    P(x_{k-1},m_k|Z_{0:k-1},U_{0:k-1},x_0)
\end{equation}
Za predpokladu, že proces prechodu stavov je Markovov proces, je možné \acrshort{slam} určiť Bayersovským odhadom pozostávajúcim z krokov predikcie a korekcie.\cite{fs2}
\begin{equation}
    \begin{split}
        & P(x_k,m_k|Z_{0:k-1},U_{0:k},x_0)=\int P(x_k|x_{k-1},u_k) \cdot P(x_{k-1},m_k|Z_{0:k-1},U_{0:k-1},x_0)dx_{k-1}\\
        & P(x_k,m_k|Z_{0:k},U_{0:k},x_0)=\frac{P(z_k|x_k,m_k)\cdot P(x_k,m_k|Z_{0:k-1},U_{0:k},x_0)}{P(z_k|Z_{0:k-1},U_{0:k})}
    \end{split}
\end{equation}
Kde:
\begin{itemize}
    \item $P(x_k|x_{k-1},u_k)$ je model systému
    \item $P(z_k|x_k,m_k)$ je model merania
\end{itemize}\newpage

\subsection{Model systému}
Zostavenie modelu pre systém Artaban vo forme:
\begin{equation}
    x_{i}=x_{i-1}u_iA
\end{equation}
Kde:
\begin{itemize}
    \item $x_i$ je stavový vektor novej polohy
    \item $x_{i-1}$ je stavový vektor predošlej polohy
    \item $u_i$ je vektor akčných zásahov
    \item $A$ je matica funkcií popisujúcich vzťahy medzi $u_i$ a $x_{i-1}$ 
\end{itemize}
Je časovo extrémne náročné riešenie, vyžadujúce popis vysoko nelineárnych, medzi sebou prepojených vzťahov. Linearizáciou týchto vzťahov môže vzniknúť model, ktorého vlastnosti pravdepodobne nebudú schopné dostatočne presne opísať pohyb robota Artaban.\newline
Podľa tvrdení tímu Panza Robotics, \cite{est} a \cite{propEst} je výhodné kontinuálne odhadovať pohybový model pomocou pozorovaťeľov a Kalman flitrov. Artaban odhaduje polohu trupu pomocou lineárneho proprioceptívneho Kalman filtra
\begin{center}
    Čakám na response z Panzy lebo ich verzia je kúsok dosť iná ako \cite{propEst} taaak hádam čo to dostanem
\end{center}

\newpage\subsubsection{Overovanie pozorovateľa polohy robota}
Proprioceptívny pozorovateľ robota Artaban bol overený simulačne ale nie v podmienkach reálneho asveta. Preto sme sa rozhodli overiť pozorovateľ nasadení na zariadení Artaban V 0.2. Pre overenie presnosti a funkčnosti pozorovateľa bol použitý systém Optitrack.
\begin{figure}[!htbp]
    \centering
    \includegraphics[width=6cm]{img/a1_meas.png}
    \caption{Artaban V 0.2 pripravený na meranie}
    \label{artMeas}
\end{figure}	

\noindent Značky systému Optitrack boli umiestnené nasledovne:
\begin{center}
    W.I.P lebo neni foto XD
\end{center}
Značky sme prvotne umiestnili na chrbát, hlavu a okolo trupu. Toto prvotné riešenie nebolo použiteľné, kvôli nepresným natočeniam osí $x$, $y$ a $z$ v softvéri Motive. Smery bolo potrebné manuálne ladiť, čo sme zhodnotili za nepoužiteľné.\newline
Preto sme značky umiestnili len na chrbát. Motive omnoho lepšie odhadol smery osí a prenos chyby manuálnej kalibrácie bol minimalizovaný.

\newpage\subsubsection{Overovanie pozorovateľa polohy robota: Nastavenie experimentov}
Robot bol postavený na snímanú plochu. Prebehlo spustenie odometrie robota a vynulovanie polohy robota a systému Optitrack.\newline
Následne sa robot pohyboval náhodne po snímanej ploche. Pohyby a trvanie experimentu boli vždy iné.

\subsubsection{Overovanie pozorovateľa polohy robota: Experiment 1}
\begin{center}
    Grafy treba prerobiť taakže tie pôjdu sem potom
\end{center}
Štatistické ukazovatele:
\begin{itemize}
    \item Prejdená vzdialenosť (Optitrack): 13.042538 [m]
    \item Prejdená vzdialenosť (Robot): 17.366829 [m]
    \item RMSE prejdenej vzdialenosti: 2.535 [m]
    \item RMSE uhol rolovania (roll): 0.163926 [°]
    \item RMSE uhol sklonu (pitch): 0.085971 [°]
    \item RMSE uhol vybočenia (yaw): 3.747096 [°] 
    \item RMSE os $x$: 0.062433 [m]
    \item RMSE os $y$: 0.098549 [m]
    \item RMSE os $z$: 0.078166 [m]
\end{itemize}
Je dôležité podotknúť, že zvýšená hodnota RMSE uhlu vybočenia je spôsobená osciláciou hodnoty medzi -180° a 180°, keďže \acrshort{ros} transformuje radiány na stupne v tomto rozmedzí, čo spôsobí nesprávne vyhodnotenie RMSE pri týchto krajných hodnotách.

\newpage\subsubsection{Overovanie pozorovateľa polohy robota: Experiment 2}
\begin{center}
    Grafy comming soon
\end{center}
Štatistické ukazovatele:
\begin{itemize}
    \item Prejdená vzdialenosť (Optitrack): 3.510216 [m]
    \item Prejdená vzdialenosť (Robot): 4.337198 [m]
    \item RMSE prejdenej vzdialenosti: 0.492973 [m]
    \item RMSE uhol rolovania (roll): 0.161815 [°]
    \item RMSE uhol sklonu (pitch): 0.071409 [°]
    \item RMSE uhol vybočenia (yaw): 0.018541 [°] 
    \item RMSE os $x$: 0.032619 [m]
    \item RMSE os $y$: 0.023173 [m]
    \item RMSE os $z$: 0.032046 [m]
\end{itemize}

\newpage\subsubsection{Overovanie pozorovateľa polohy robota: Experiment 3}
\begin{center}
    Grafy comming soon
\end{center}
Štatistické ukazovatele:
\begin{itemize}
    \item Prejdená vzdialenosť (Optitrack): 14.614010 [m]
    \item Prejdená vzdialenosť (Robot): 18.108734 [m]
    \item RMSE prejdenej vzdialenosti: 2.051004 [m]
    \item RMSE uhol rolovania (roll): 0.091147 [°]
    \item RMSE uhol sklonu (pitch): 0.074933 [°]
    \item RMSE uhol vybočenia (yaw): 0.149483 [°] 
    \item RMSE os $x$: 0.135738 [m]
    \item RMSE os $y$: 0.088887 [m]
    \item RMSE os $z$: 0.036152 [m]
\end{itemize}